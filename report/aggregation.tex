\documentclass{article}
% load packages
\usepackage{amsmath}
\usepackage{amssymb}
\usepackage{hyperref}
% setup bibliography
\usepackage[style=authoryear,backend=biber]{biblatex}
\addbibresource{aggregation.bib}
% ----------------------------- %
\title{Aggregation by Movement}
\author{Ole Schügl}
% ----------------------------- %
\begin{document}
\maketitle

\section{Introduction}
When creating realistic models, it is important to be explicit about the assumptions and why they are justified.
An implicit assumption when describing populations in terms of ODEs is often the uniform distribution of indiviuals in space. (Lotka-Volterra, Logistic, ...)
\autocite{durrettImportanceBeingDiscrete1994} showed that local effects can have qualitative impacts on the dynamics of a system, for instance allowing the coexistence of two species where ODEs predict a single species to survive.

Since then, a large number of studies have shown how spatial patterns can improve the stability of ecosystems and mechanisms for how these patterns can form.
Many of these explain spatial patterns in terms of spatial variation in births and deaths of individuals \autocite{liuPhaseSeparationDriven2016} (not primary source) and are based on the activation-inhibition principle developed by Turing and Meinhardt (cite).
A different process enabling pattern formation is based on movement based on a principle called phase-separation. 

Spatial patterns form because of density-dependent movement while the population stays constant. 
This phenomenon has been observed in mussel-beds, ants, slime moulds and bacteria \autocite{liuPhaseSeparationDriven2016}.
In this report, I will present a simple model that exhibits self-organized spatial patterns based on density-dependent movement.

\section{Preliminary Calculations: A Single Organism}
Before adding spatial interactions, a single organism diffusing in space will be analysed.
This will act as a baseline and can be compared to the more complicated case.

To do this, we consider a simple environment with only a single organism and a constant diffusion term. 
Additionally, it will be assumed that the environment is infinite. 
The simulations will be run on a torus later, but that would make an analytical derivation more complicated.\\
The walker position will change according to the following equations:
\begin{align}
    & x(t + dt) = x(t) + \sqrt{2D dt} u_x \label{single_x}\\
    & y(t + dt) = y(t) + \sqrt{2D dt} u_y
\end{align}
where $u_x, u_y$ are independent random numbers $\sim \mathcal{N}(0,1)$.\\
Let's consider the increments $dx(t+dt) = x(t + dt) - x(t)$.
Then 
\begin{equation*}
    x(t) = \sum_{\tau = 0}^{t/dt} dx(\tau) + x(0),
\end{equation*}
that is, the position of $x$ at time $t$ can be described by adding up the increments.
From \ref{single_x}, we can see that the increments are normally distributed with variance $2D dt$.
The distribution of the displacements $x(t) - x(0)$ is therefore a sum of normally distributed random variables, thus being again normally distributed with the variance being the sum of the variances 
\begin{equation*}
    x(t) - x(0) \sim \mathcal{N}(0,\sum 2D dt) =  \mathcal{N}(0,2Dt).
\end{equation*}
To make the following equations clearer, we will use $\tilde{x}(t) = x(t)-x(0)$ and $\tilde{y}(t) = y(t)-y(0)$ .
Using this, we calculate the mean squared displacement 
\begin{equation*}
    MSD(t) = \langle \tilde{x}^2(t) +\tilde{y}^2(t) \rangle = \langle \tilde{x}^2(t)\rangle + \langle\tilde{y}^2(t) \rangle.
\end{equation*}
Again we focus on the term involving $x$.
\begin{align*}
    \langle \tilde{x}^2(t)\rangle = \int_{-\infty}^{\infty} \tilde{x}^2 p(\tilde{x},t)d\tilde{x} = \int_{-\infty}^{\infty} \tilde{x}^2 \frac{1}{\sqrt{4\pi Dt}} e^{-\frac{\tilde{x}^2}{4Dt}}d\tilde{x}
\end{align*}
Solving this integral requires a few steps, that will be only briefly mentioned. 
First, we do a variable transformation by setting $y = \frac{\tilde{x}}{\sqrt{4Dt}}$. 
The integral then becomes
\begin{equation*}
    \frac{4Dt}{\sqrt{\pi}}\int_{-\infty}^{\infty} y^2  e^{-y^2}dy.
\end{equation*}
This integral can be solved by partial integration and is equal to $\sqrt{\pi}/2$.
Thus, the final result is 
\begin{equation*}
    \langle \tilde{x}^2(t)\rangle = 2Dt.
\end{equation*}
For $\langle \tilde{y}^2(t)\rangle$, we get the exact same result, and the mean squared displacement is therefore
\begin{equation*}
    MSD(t) = \langle \tilde{x}^2(t)\rangle + \langle \tilde{y}^2(t)\rangle  = 4Dt.
\end{equation*}
The average absolute distance of the organism from the initial position therefore grows with the square root of time.

\section{Aggregation}
Now, we will add 
\section{Results}

\section{Discussion}
\printbibliography
\end{document}